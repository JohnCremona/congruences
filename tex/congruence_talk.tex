%
%  The symplectic type of congruences between elliptic curves
%
%  Talk first given at
%
%  AGC2T, Luminy June 2019
%
%  Time-stamp: <2019-06-07 15:04:32 masgaj>
%
\documentclass[compress]{beamer}
%\documentclass[handout]{beamer}

% for themes, etc.
   \usetheme{}
\mode<presentation>%<handout> 
%{ \useinnertheme{rounded}
  \setbeamerfont{block title}{size={}}
  \setbeamertemplate{navigation symbols}[only frame symbol]
  \usefonttheme[onlymath]{serif}
  \useoutertheme[footline=empty]{miniframes} %,subsection=false
%}

\usepackage{comment}
\usepackage{times}
\usepackage{graphicx}
\usepackage{amsmath,amssymb,amsthm,bbm} %,calc,picins}
\usepackage[mathcal]{euscript}
\usepackage{mathrsfs}
\usepackage{booktabs}
\usepackage{numprint}

\setlength{\parindent}{0pt}

\newtheorem{remark}{Remark}
%% \newtheorem{theorem}{Theorem}
%% \newtheorem{lemma}{Lemma}
\newtheorem{prop}{Proposition}
\newtheorem{cor}{Corollary}


% ---------------------------------
\renewcommand{\geq}{\geqslant}
\renewcommand{\leq}{\leqslant}

\newcommand{\Q}{\mathbb Q}
\newcommand{\Qbar}{\overline{\Q}}
\newcommand{\Kbar}{\overline{K}}
\newcommand{\rhobar}{\overline{\rho}}
\newcommand{\F}{\mathbb F}
\newcommand{\Z}{\mathbb Z}
\newcommand{\R}{\mathbb R}
\newcommand{\OO}{\mathcal O}
\newcommand{\CC}{\mathcal C}
\newcommand{\LL}{\mathcal L}
\newcommand{\C}{\mathbb C}
\newcommand{\PP}{\mathbb P}
\newcommand{\Fp}{\mathbb F_{\! p}}
\newcommand{\sss}{\scriptscriptstyle}
\newcommand{\idealin}{\vartriangleleft}
\newcommand{\GL}{\operatorname{GL}}
\newcommand{\SL}{\operatorname{SL}}
\newcommand{\PGL}{\operatorname{PGL}}
\newcommand{\PSL}{\operatorname{PSL}}
\newcommand{\Gal}{\operatorname{Gal}}
\newcommand{\End}{\operatorname{End}}
\newcommand{\n}{\mathfrak n}
\newcommand{\p}{\mathfrak p}
\newcommand{\q}{\mathfrak q}


\DeclareMathOperator{\ord}{ord}
\DeclareMathOperator{\im}{im}
\DeclareMathOperator{\rank}{rank}
\DeclareMathOperator{\Reg}{Reg}
\DeclareMathOperator{\Pf}{Pf}
%\DeclareMathOperator{\rank}{rank}

%--   Colours
\definecolor{dgreen}{rgb}{0,0.4,0.4}
\definecolor{dblue}{rgb}{0,0,0.7}
\definecolor{dred}{rgb}{0.7,0,0}
\definecolor{lightgrey}{rgb}{0.9,0.9,0.9}
\setbeamercolor{sagecolors}{fg=dblue,bg=lightgrey}


\DeclareMathOperator{\disc}{disc}
\DeclareMathOperator{\cond}{cond}
\DeclareMathOperator{\Sel}{Sel}
\newcommand\Sage{{\sc Sage}}
\newcommand\Magma{{\sc Magma}}
\newcommand{\BF}[1]{{\bf #1:}\hspace{1em}\ignorespaces}
\newcommand{\lmfdbec}[3]{\color{blue}\href{http://www.lmfdb.org/EllipticCurve/Q/#1#2#3}{{\text{\rm#1#2#3}}}}
\newcommand{\LMFDB}{\href{http://www.lmfdb.org/EllipticCurve/Q}{\color{blue}LMFDB}}
\newcommand{\high}[1]{\emph{\color{blue}{#1}}}

% titlepage -----------------------

\title
{
The symplectic type of congruences between elliptic curves
}

\author{\href{https://warwick.ac.uk/fac/sci/maths/people/staff/john_cremona/}{John Cremona}}

\institute{University of Warwick\\---\\ joint work with
  \href{https://homepages.warwick.ac.uk/staff/Nuno.Freitas/}{Nuno Freitas}
  (Warwick) %\includegraphics[height=1cm]{nuno.pdf}
}

\date{AGC${}^2$T Luminy, 10 June 2019}

% --------------------------------
\begin{document}

\frame[plain]{
 \maketitle
\vskip-40pt
\hbox to \hsize{
 \includegraphics[height=10mm]{./epsrc.pdf}
\hfill
 \includegraphics[height=10mm]{./eu_flag.pdf}
\hfill
 \includegraphics%[height=10mm]{./UW-blue.pdf}
[scale=0.2]{UofW_CMYK_Colour_logo_+Descriptor.pdf}}
}


\begin{frame}\frametitle{Overview}
 \begin{enumerate}
\item Elliptic curves, mod~$p$ Galois representations, Weil pairing.
\item Congruences between curves, symplectic types.  The isogeny criterion.
\item The Frey--Mazur Conjecture over~$\Q$. % Theorem.
\item Finding all congruences in the \LMFDB\ database. % Results.
\item Determining the symplectic type using modular curves
  ($p=7$). % Results.
\item Congruences between twists.
 \end{enumerate}


\end{frame}

\begin{frame}\frametitle{Elliptic Curves}
In this talk we consider elliptic curves over a number field~$K$, for
example ~$K=\Q$.

If we need explicit equations we'll use short Weierstrass models
\[  E_{a,b}:\quad Y^2 = X^3+aX+b
\]
with $a,b\in K$ such that $4 a^{3} + 27 b^{2} \not=0$. \pause

The set of $K$-rational points $E(K)$ forms an abelian group.

For $m\ge2$ we denote by $E[m]$ the \high{$m$-torsion subgroup}:
\[
  E[m] = \{ P \in E(\Kbar) \mid mP=0 \}.
  \]
  \pause
  We have $E[m]\cong (\Z/mZ)^2$ as abelian groups.
  \medskip
  
  But $E[m]$ carries additional structure\dots.
\end{frame}

\begin{frame}\frametitle{Mod $p$ Galois representations}
  Let $G_K=\Gal(\Kbar/K)$, the \high{absolute Galois group} of~$K$.
  This acts on $E(\Kbar)$ by acting on coordinates:
  \[
   P=(x,y)\in E(\Kbar), \quad\sigma\in G_K:\qquad
   \sigma(P)=(\sigma(x),\sigma(y))\in E(\Kbar).
   \]
   \pause
 The Galois action preserves the group structure:
   \[
   \sigma(P+Q)=\sigma(P)+\sigma(Q).
   \]
   Hence each $E[m]$ is a \high{$G_K$-module}.
   \pause\medskip

   Taking $m=p$ prime, $E[p]$ is a $2$-dimensional vector space
   over~$\F_p$.  Fixing a basis of $E[p]$ we obtain the \high{mod~$p$
   Galois representation}
   \[
   \rhobar_{E,p}: G_K \to \GL_2(\F_p).
   \]

\end{frame}

\begin{frame}\frametitle{The Weil pairing}
  As well as its vector space structure, $E[p]$ admits a \high{symplectic
  structure}: there is a non-degenerate alternating bilinear pairing,
  the \high{Weil pairing}
  \[
  e_p = e_{E,p}:\quad E[p]\times E[p] \to \mu_p
  \]
  where~$\mu_p$ denotes the group of $p$'th roots of unity in
  $\Qbar^*$.
  \pause\medskip

  The Weil pairing is \high{Galois equivariant}:
  \[
  e_p(\sigma(P),\sigma(Q)) = \sigma(e_p(P,Q)) = e_p(P,Q)^{\chi_p(\sigma)}
  \]
  where $\chi_p:G_K\to\F_p^*$ is the cyclotomic character.

  \pause\medskip This Galois-equivariant symplectic structure on $E[p]$ is
  what we are interested in.
\end{frame}

\begin{frame}\frametitle{Congruences and their symplectic types}
  We are interested in the situation where two different curves have
  \high{isomorphic} $p$-torsion modules.
  \pause\medskip

  $E_1$ and $E_2$ are said to satisfy a \high{mod~$p$ congruence} if
  there is a map
  \[
  \phi:\quad E_1[p] \to E_2[p]
  \]
  which is both $\F_p$-linear and $G_K$-equivariant, \textit{i.e.}, is
  an isomorphism of $G_K$-modules.
  \pause\medskip

  To each such~$\phi$ there is a constant $d_{\phi}\in\F_p^*$ such
  that
  \[
  e_{E_2,p}(\phi(P),\phi(Q)) = e_{E_1,p}(P,Q)^{d_{\phi}}.
  \]

  We say that $\phi$ is \high{symplectic} if $d_{\phi}$ is a
  \high{square} mod~$p$ and \high{antisymplectic} otherwise.
\end{frame}

\begin{frame}\frametitle{Pedantic remarks}
  \begin{itemize}
    \item This is uninteresting for $p=2$ so we'll assume that $p$ is
      \high{odd}.

    \item The Weil pairing is not well-defined by the properties
      given.  If $E[p]$ has $G_K$-automorphisms we can compose with
      $e_p$ to get a new pairing which will equal $e_p^k$ for some
      $k\in\F_p^*$.

      But if $E[p]$ is \high{irreducible} as
      $G_K$-module then the only such automorphisms are scalar
      multiplications, for which $d$ is a square.  We will only talk
      about the symplectic type of isomorphisms between irreducible
      modules: this \high{is} well-defined.
  \end{itemize}
\end{frame}

\begin{frame}\frametitle{Isogenies}
  Isogenies between curves provide one source of congruences.
  \pause\medskip

  Let $\phi:E_1\to E_2$ be an isogeny of degree~$\deg(\phi)$
  \high{coprime to $p$}, defined over~$K$.  Then $\phi$ induces an
  $\F_p$-isomorphism $E_1[p]\to E_2[p]$.  The \high{isogeny criterion}
  says that $\phi$ is symplectic if and only if the Legendre symbol
  $({\deg(\phi)}/{p})=+1$.
  \pause\medskip

  \begin{proof}
    Using Weil reciprocity,
    \[
    \begin{aligned}
      e_{E_2,p}(\phi(P), \phi(Q)) &= e_{E_1,p}(P, \hat\phi\phi(Q)) \\
      &= e_{E_1,p}(P, \deg(\phi)(Q))\\ &= e_{E_1,p}(P,
      Q)^{\deg(\phi)},
    \end{aligned}
  \]
  where $\hat{\phi}$ denotes the dual isogeny, since $\hat\phi\phi=\deg(\phi)$.
  \end{proof}
  \pause\medskip

  Do any other mod~$p$ congruences exist?
\end{frame}

\begin{frame}\frametitle{The Frey-Mazur conjecture}
  The \high{Frey--Mazur conjecture} (over $\Q$) states:

  \begin{quote}There is a constant $C$ such that, if $E_1/\Q$ and $E_2/\Q$
  satisfy $E_1[p] \simeq E_2[p]$ as $G_\Q$-modules for some prime $p >
  C$, then $E_1$ and $E_2$ are $\Q$-isogenous.
  \end{quote}

  \pause\medskip

  \begin{theorem}[C. \& Freitas]
  If $E_1$ and~$E_2/\Q$ both have conductor ${}\le\numprint{400000}$
  are not isogenous, and satisfy $E_1[p] \simeq E_2[p]$ as
  $G_\Q$-modules for some prime $p$, then $p\le17$.
  \end{theorem}

  \pause\medskip
  \begin{itemize}
    \item It is generally believed that the Frey--Mazur
      conjecture is true, with $C=17$. More on this later later.
      \item Congruences for small~$p$ are common; for $p=17$ there is
        essentially only one known, between $\lmfdbec{47775}{b}{1}$
        and~$\lmfdbec{3675}{b}{1}$.
  \end{itemize}

\end{frame}

\begin{frame}\frametitle{Finding congruences in the LMFDB database}
The \LMFDB\ database contains all elliptic curves defined over~$\Q$ of
conductor up to $\numprint{400000}$: that is $\numprint{2483649}$
curves in $\numprint{1741002}$ isogeny classes.

What congruences are there between (non-isogenous) curves, and how do
we find them?

Two representations have isomorphic semisimplifications if and only if
they have the same traces. We can test this condition by testing
whether
\[ a_{\ell}(E_1)\equiv a_{\ell}(E_2)\pmod{p}
\quad \text{for all primes } \ell \nmid pN_1N_2,
\] 
where $N_1$ and $N_2$ are the conductors of~$E_1$ and $E_2$.

But there are infinitely many primes~$\ell$.  And each for curve we need to
ignore a different bad set!
\end{frame}

\begin{frame}\frametitle{Sieving}
  To get around these issues we use a \high{sieve} with a hash
  function, and only test $\ell>\numprint{400000}$.

  Let $\LL_B=\{\ell_0,\dots,\ell_{B-1}\}$ be the set of the $B$
  smallest primes greater than $\numprint{400000}$.  For each $p$ we
  define the hash of $E$ to be
  \[\sum_{i=0}^{B-1}\overline{a}_{\ell_i}(E)p^i\in\Z.\]

 Any two $p$-congruent curves (up to semisimplification) have the same
 hash value.  If $B$ is not too small then we will get few (if any)
 ``false positive'' clashes.

  We can also parallelise this with repect to~$p$, so that we only
  need to compute each $a_{\ell}(E)$ once.  Against each hash value,
  we store lists of curves which have that $p$-hash (processing the
  curves one at a time, one from each isogeny class).  At the end we
  extract the lists of size at least~$2$, to give us sets of curves
  which are likely to all be $p$-congruent. % (up to  semisimplification).

\end{frame}

\begin{frame}\frametitle{Sieving in practice}
  This works well in practice with $B=40$. Not quite with $B=35$!

  \pause
  \medskip
  The curves with labels $\lmfdbec{25921}{a}{1}$ and
  $\lmfdbec{78400}{gw}{1}$ have traces $a_{\ell}$ which are \high{equal
    for all~$\ell\in\LL_{35}$}, that is, for all~$\ell$ with
  $400000\le \ell<400457$ (though not for the 36th $\ell=400457$).
  \pause
  \medskip

  Note on reducibility: here we are testing for $p$-congruence only up
  to semisimplification.  For curves with $E[p]$ reducible
  (\textit{i.e.}, which have a rational $p$-isogeny) this is a weaker
  condition than $p$-congruence, and we need to carry out further
  tests.

    \pause
  \medskip
  We also need to test whether curves which appear to be $p$-congruent
  after sieving actually are.  With $B=40$ this is always the case.
\end{frame}

\begin{frame}\frametitle{Sieving results}
  For $5\le p\le 97$ we find the following number of sets of more than
  one mutually $p$-congruent curves (up to semisimplification,
  ignoring isogenies):

  %% p=5: 102043 subsets (101717 irred, 326 red) sizes up to 18
  %% (irred), 430 (red)
  %% p=7: 20138 subsets (19883 irred, 255 red) sizes up to 5 (irred),
  %% 76 (red)
  %% p=11: 635 subsets, size 2, all irred
  %% p=13: 150, size 2, all irred
  %% p=17: 8, size 2, all irred
  %% p>=19: none
  \[
  \begin{tabular}{||c||c|c|c|c|c||}
    \hline
    $p$ & \#sets & \# irred. & max.irred. & \# red. & max. red.\\
    \hline
    5 & 102043 & 101717 & 18 & 326 & 430 \\
    7 & 20138 & 19883 & 5 & 255 & 76 \\
    11 & 635 & 635 & 2 & 0 & - \\
    13 & 150 & 150 & 2 & 0 & - \\
    17 & 8 & 8 & 2 & 0 & - \\
    19$\le p\le$ 97 & 0 & 0 & - & 0 & - \\
    \hline
  \end{tabular}
  \]

After eliminating reducibles which are not isomorphic, for $p=7$ we
find $337$ non-trivial sets, of size up to~$4$.
\end{frame}

\begin{frame}\frametitle{Distinguishing symplectic from
    antisymplectic} Freitas and Kraus have a long paper which gives
  many different \high{local} criteria for determining whether a
  congruence $E_1[p]\cong E_2[p]$ is symplectic or antisymplectic.
  These criteria are not guaranteed to apply in all cases.

\pause\medskip

  For $p=7$ we use a method based on \high{modular curves}.

  Recall that for each prime $p$ there is a modular curve $X(p)$
  defined over $\Q$ which parametrises elliptic curves with a fixed
  level structure, \textit{i.e.}, a fixed structure of $E[p]$ as a
  Galois module.  For $p=7$ this curve has genus $3$, and the Klein
  quartic is one model for it.

    \pause
  \medskip
  Fix one elliptic curve over $\Q$.  Then there exists a curve
  $X_E(p)$, which is a twist of $X(p)$, whose (non-cuspidal) points
  correspond to curves $E'$ with $E[p]\cong E'[p]$ symplectically.
  (Strictly, to pairs $(E',\alpha)$ where $\alpha:E[p]\to E'[p]$ is a
  symplectic isomorphism, up to scaling.)
\end{frame}

\begin{frame}\frametitle{$X_E^{\pm}(p)$}
Similarly there is another twist $X_E^-(p)$ parametrizing curves $E'$
which are antisymplectically isomorphic to~$E$.
    \pause
  \medskip

An explicit model for $X_E(7)$ was found by Kraus and Halberstadt
(2003) together with the degree~$168$ map $j:X_E(7)\to X(1)=\PP^1$
(giving the $j$-invariant of the congruent curve $E'$, and incomplete
formulas for the coefficients of $E'$.

    \pause
  \medskip
More complete formulas were provided by Fisher (2014), who also gave
all the formulas for $X_E^-(7)$ parametrizing antisymplectic
congruences, and $X_E^{\pm}(11)$ (which has genus~$26$).

    \pause
  \medskip
For $p=7$ we implementated these formulas and apply them as follows.
\end{frame}

\begin{frame}\frametitle{Using $X_E^{\pm}(7)$: the algorithm}
\begin{itemize}
\item Given two elliptic curves $E$, $E'$ defined over a field~$K$ of
  characteristic~$0$. We do not need to assume anything about them.
  Compute $j(E')$.
\item Use the explicit map $j:X_E(7)\to\PP^1$ to find the preimages
  (if any) of $j(E')$ in $X_E(7)(K)$.  If none then $E,E'$ are not
  symplectically $p$-congruent over~$K$.
\item For any $P\in X_E(7)(K)$ use Fisher's formulas to find a model
  for the associated congruent curve $E''$.
\item If $E'\cong E''$ for any of these, then $E,E'$ are $p$-congruent
  over $K$, otherwise not.
\item repeat with $X_E^-(7)$.
\end{itemize}
\end{frame}

\begin{frame}\frametitle{Using $X_E^{\pm}(7)$: results}
Of the $\numprint{19883}$ non-trivial sets of isogeny classes with mutually
isomorphic irreducible mod~$7$ representations, we find that in
$\numprint{12394}$ cases all the isomorphisms are symplectic, while in the
remaining $\numprint{7489}$ cases antisymplectic isomorphisms occur.

We checked in all cases where the local criteria apply that the result
are consistent, and they were.  In fact the local criteria were able
to decide all the cases from the database (but the local criteria were
expanded for this to be the case).

For example, there are $5092$ pairs of curves with irreducible mod~$7$
representations which are antisymplectically congruent.  Of these, the
local tests work in $5090$ cases but not the remaining two.

Of $411$ pairs of $7$-congruent reducible pairs, local test work in
$245$ cases but not the remaining $166$.
\end{frame}

\begin{frame}\frametitle{Results for $p>7$}
For $p\ge11$ we used the local criteria only to test congruences.

It would be possible to implement Fisher's formulas for
$X_E^{\pm}(11)$, but we have not yet done so.

For $11\le p\le 17$ we only find congruences with $E[p]$ irreducible
and we never find sets of more than three congruent curves (excluding
isogenies).


  \[
  \begin{tabular}{||c||c|c|c||}
    \hline
    $p$ & \# congruent pairs & \# symplectic & \# antisymplectic\\
    \hline
    11 & 635 & 446 & 189 \\
    13 & 150 & 88 & 62 \\
    17 & 8 & 0 & 8 \\
    \hline
  \end{tabular}
  \]
%% for $p=11$, of the $635$ congruent pairs of isogeny classes, $446$ are
%% symplectic and $189$ are antisymplectic; for $p=13$, of the $150$
%% congruent pairs of isogeny classes, $88$ are symplectic and $62$ are
%% antisymplectic; for $p=17$, all of the $8$ congruent pairs of isogeny
%% classes are antisymplectic.
\end{frame}

\begin{frame}\frametitle{Twists}
As well as these computational results, we also have various results
of a more theoretical nature.  Many of these involve \high{twists}.

First, it is easy to show that when we have a congruence $E_1[p]\cong
E_2[p]$ then for any quadratic twist (associated to a quadratic
extension $K(\sqrt{d})/K$), the twisted curves also satisfy a
$p$-congruence: $E_1^d[p]\cong E_2^d[p]$.  Moreover the symplectic
type is preserved except in one very special situation.

So perhaps the previous tables should have only shown the number of
congruences ``up to twist''.  This is not so easy, since twisting
changes the conductor in general.  However, we can count the total
number of curves, up to twist, appearing in any of the congruences we
found:

For $p=7$ there are $\numprint{10348}$ distinct $j$-invariants of
curves with irreducible mod~$7$ representations which are congruent to
at least one non-isogenous curve, and $358$ distinct $j$-invariants in
the reducible case.

For $p=11$ there are $191$ distinct $j$-invariants and for $p=13$
there are $39$.  For $p=17$, all $17$-congruent isogeny classes
consist of single curves, the eight pairs are quadratic twists, and
the $j$-invariants of the curves in each pair are
$48412981936758748562855/77853743274432041397$ and $-46585/243$.  One
such pair of $17$-congruent curves consists of $\lmfdbec{47775}{b}{1}$
and~$\lmfdbec{3675}{b}{1}$.

\end{frame}

\begin{frame}\frametitle{}
\end{frame}

\begin{frame}\frametitle{}
\end{frame}

\begin{frame}\frametitle{}
\end{frame}

\begin{frame}\frametitle{}
\end{frame}

\begin{frame}\frametitle{Using modular curves} % for congruences and their symplectic type}
\end{frame}

\begin{frame}\frametitle{Some results on congruences between twists}
\end{frame}



\end{document}
