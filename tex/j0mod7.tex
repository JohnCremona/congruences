%  FILE J0MOD7.TEX
%
%
%  Time-stamp: <2018-07-10 10:04:04 masgaj>
%

\documentclass[a4paper, 10pt]{amsart}

\usepackage{amsmath,amsfonts,amsthm,amssymb}
\usepackage{url}
\usepackage{hyperref}
\usepackage{comment}
\newtheorem{thm}{Theorem}[section]
\newtheorem{cor}[thm]{Corollary}
\newtheorem{lem}[thm]{Lemma}
\newtheorem{prop}[thm]{Proposition}
\newtheorem{defn}{Definition}%[section]
\newtheorem{rem}[thm]{Remark}

\hfuzz1pc % Don't bother to report overfull boxes if overage is < 1pc

\def\Z{{\mathbb Z}}
\def\Q{{\mathbb Q}}
\def\R{{\mathbb R}}
\def\C{{\mathbb C}}
\def\T{{\mathbb T}}
\def\H{{\mathbb H}}
%\def\RC{{K_\infty}}
\def\RC{{\tilde{K}}}
\def\P{{\mathbb P}}
\def\F{{\mathbb F}}
\def\Fp{{\mathbb F}_p}
\def\Fq{{\mathbb F}_q}
\def\M{{\mathcal M}}
\def\O{{\mathcal O}}
\def\CC{{\mathcal C}}
\def\HH{{\mathcal H}}
\def\SS{{\mathcal S}}
\def\G{{\mathcal G}}
\def\K{{\mathcal K}}
\def\I{{\mathcal I}}
\def\J{{\mathcal J}}
\def\PP{{\mathcal P}}
\def\ZG{{\mathcal Z}}
\def\a{{\mathfrak a}}
\def\b{{\mathbf b}}
\def\v{{\mathbf v}}
\def\c{{\mathfrak c}}
\def\d{{\mathfrak d}}
\def\m{{\mathfrak m}}
\def\n{{\mathfrak n}}
\def\p{{\mathfrak p}}
\def\q{{\mathfrak q}}
\def\r{{\mathfrak r}}
\def\DF{{\nabla\underline{F}}}
\def\Abar{\overline{A}}
\def\RR{{R\oplus R}}
\def\KK{{K\oplus K}}
\def\uF{{\underline{F}}}
\def\oC{\overline{\mathcal C}}
\def\oP{\overline{P}}
\def\oM{\overline{M}}
\DeclareMathOperator{\lcm}{lcm}
\DeclareMathOperator{\disc}{disc}
\DeclareMathOperator{\rk}{rank}
\DeclareMathOperator{\ord}{ord}
\DeclareMathOperator{\diag}{diag}
\DeclareMathOperator{\adj}{adj}
\DeclareMathOperator{\Mat}{Mat}
\DeclareMathOperator{\Hom}{Hom}
\DeclareMathOperator{\Res}{Res}
\DeclareMathOperator{\GL}{GL}
\DeclareMathOperator{\SL}{SL}
\DeclareMathOperator{\PGL}{PGL}
\DeclareMathOperator{\PSL}{PSL}
\DeclareMathOperator{\U}{U}
\DeclareMathOperator{\SU}{SU}
\DeclareMathOperator{\OO}{O}
\DeclareMathOperator{\SO}{SO}
\DeclareMathOperator{\Cl}{Cl}
\def\<#1>{\left<#1\right>}
\newcommand{\mat}[4]{\left(\begin{matrix} %
                   #1 & #2 \\ #3 & #4 %
                  \end{matrix}\right)}
\newcommand{\smat}[4]{\left(\begin{smallmatrix} %
                   #1 & #2 \\ #3 & #4 %
                  \end{smallmatrix}\right)}
\DeclareMathOperator{\im}{im}
\DeclareMathOperator{\id}{id}
\def\w{\omega}
\def\r3{\sqrt{-3}}
\def\pibar{\overline{\pi}}
\def\Qbar{\overline{\Q}}
\DeclareMathOperator{\Gal}{Gal}
\DeclareMathOperator{\tw}{tw}
\def\legover{\overwithdelims()}
\def\legendre#1#2{\displaystyle{#1\legover#2}}

\begin{document}

\title{Mod-$7$ isomorphisms between elliptic curves with
  $j$-invariant~$0$}

\author{J. E. Cremona}
\address{Mathematics Institute, University of Warwick, Coventry CV4 7AL, UK}
\email{John.Cremona@gmail.com.uk}
\date{\today}

\begin {abstract}
We show that for every elliptic curve~$E$ with $j(E)=0$, there is a
non-trivial sextic twist of~$E$ which has symplectically isomorphic
mod-$7$ Galois representation.
\end {abstract}

\maketitle

\section{Introduction}
In \cite{Halberstadt-Kraus-XE7}, Halberstadt and Kraus give an
explicit model for the modular curve $X_E(7)$, for any elliptic curve
$E$ defined over a field~$K$ of characteristic not equal to $2$, $3$
or~$7$. The $K$-rational points on $X_E(7)$ parametrise pairs
$(E',\phi)$ where $E'$ is an elliptic curve defined over~$K$ and
$\phi:E[7]\to E'[7]$ is an isomorphism of symplectic $G_K$-modules.
Here the symplectic structure on $E[7]$ is given by the Weil pairing,
and we identify two such isomorphisms~$\phi$ when one is a scalar
multiple of the other.

The model for $X_E(7)$ given in \cite{Halberstadt-Kraus-XE7} is a
plane quartic curve (a twist of $X(7)$, which has the classical Klein
quartic as a model over~$\Q$) given by an explicit ternary quartic
form~$F_{a,b}(X,Y,Z)$ in $\Z[a,b][X,Y,Z]$ where $E$ has equation
$Y^2=X^3+aX+b$.  The $24$ flexes on $X_E(7)$ are the cusps; in other
words they are the poles of the degree~$168$ map to the $j$-line, $j:
X_E(7)\to X(1)$.  The $28$ bitangents intersect $X_E(7)$ at the 56
points above $j=0$: these are the intersection points of $X_E(7)$ and
the curve of degree~$14$ defined by a covariant of~$F_{a,b}$.

The base point $P_E=[0:1:0]\in X_E(7)(K)$ corresponds to the pair
$(E,\id)$.  The tangent line to $X_E(7)$ at~$P_E$ is $Z=0$.  In
general, when $a\not=0$, this line intersects $X_E(7)$ in two further
distinct points $P_{\pm}=[\pm\sqrt{-3a}:a:0]$ which are rational if
and only if $-3a\in K^{*2}$.  When this holds, the isomorphic curves
$E'$ may be found using the two parametrized families in the earlier
paper \cite{Halberstadt-Kraus-YE7} of Halberstadt and Kraus.  However,
these parametrization fail when $a=0$, that is for curves with
$j$-invariant~$0$.  The purpose of this note is to fill this gap.

When $a=0$, the second point of intersection of the bitangent $Z=0$
with $X_E(7)$, in addition to $P_E=[0:1:0]$, is $P_0=[1:0:0]$.  In
what follows we do not need to use the curve $X_E(7)$, but it is the
existence of this second rational point on $X_E(7)$ whenever $j(E)=0$
which led us to the result here.

\section{Statement of result}
Every elliptic curve with $j=0$ has a model of the form
$E_b:\ Y^2=X^3+b$ with $b$ nonzero, and the isomorphism class of $E_b$
depends only on $b$ modulo $6$th powers.  Any two such curves are
sextic twists of each other: the sextic twist of $E_b$ by $u$ is
$E_{bu}$, which is isomorphic to $E_b$ if and only if $u$ is a $6$th
power.  Our result is the following:

\begin{thm}
For every $b\not=0$, the elliptic curves $E_b$ and $E_{-28/b}$ have
symplectically isomorphic mod-$7$ Galois representations.
\end{thm}

\begin{rem}
The same method can be used to show that $E_b$ and $E_{-27b}$ have
antisymplectically isomorphic Galois representations.  However this is
more easily seen by noting that these curves are $3$-isogenous; since
$3$ is a non-quadratic residue of~$7$, a $3$-isogeny between them
induces an antisymplectic isomorphism between their mod-$7$ torsion
modules.  In case $-3$ is a square, these curves are even isomorphic,
and the $3$-isogeny is the endomorphism $\pm\r3$, while in
general the two curves are quadratic twists.
\end{rem}

For simplicity of notation we take the base field to be $\Q$.  Little
change would be required in what follows if $\Q$ were to be replaced
by any field of characteristic different from~$2$, $3$ or $7$ in which
$-3$ is not a square.  In case $-3$ is a square there is some
simplification, and the result still holds.

\section{Proof: preliminaries}
Let $K=\Q(\r3)=\Q(\w)$ where $\w$ satisfies
$\w^2+\w+1=0$, so that $\w$ is a primitive $3$rd root of
unity and $-\w$ a primitive $6$th root of unity.  The elliptic
curves $E_b$ have CM by the maximal order $\O_K=\Z[\w]$ in which
$7$ splits as $7=\pi\pibar$ where $\pi=3\w+2$.  Set
$\r3=2w+1$ (to fix the sign).  We denote the nontrivial
automorphism of $K/\Q$ by $x\mapsto x'$.

Fix $b\in\Q^*$ defining the elliptic curve $E_b: Y^2=X^3+b$.  The
$2$-dimensional $\F_7$-vector space $E_b[7]$ has two $K$-rational
one-dimensional subspaces, namely the kernels of the endomorphisms
$[\pi]$ and $[\pibar]$, which we make explicit.

Define
\begin{align}
\alpha&=4b/\pibar,\\
\beta=\alpha+b&=-\w^2\r3^3b/\pibar,\quad\text{and}\\
\gamma&=-4\r3^3b/\pibar = -\r3^3\alpha = 4\w\beta.
\end{align}
Then for any choice of roots, the point $(\alpha^{1/3},\beta^{1/2})\in
E_b(\Qbar)$.  We fix one $6$th root $\gamma^{1/6}$ once and for all
and set
\begin{align}
\alpha^{1/3} &= \gamma^{1/3}/\r3,\\
\beta^{1/2} &= \gamma^{1/2}\w/2
\end{align}
(with $\gamma^{1/3}=(\gamma^{1/6})^2$ and $\gamma^{1/2}=(\gamma^{1/6})^3$).  Let
$P=(\alpha^{1/3},\beta^{1/2})\in E_b(\Qbar)$.

Computing the $7$-division polynomial of~$E_b$, we find that over
$\Q(b)$ (treating $b$ as an indeterminate) it has factors $X^3-\alpha$
and $X^3-\alpha'$, the remaining factor of degree~$18$ being
irreducible.  It follows that $P$ and the conjugate point
$P'=(\alpha'^{1/3},\beta'^{1/2})$ have order~$7$.  Also, the multiples
$kP$ for $1\le k\le6$ all have $x(kP)^3=\alpha$ (this can be checked
by direct calculation, or by observing that $x(2P)^3=\alpha'$ yields a
contradiction after repeating three times since $8P=P$).

Thus the $7$-division field is
$L=K(E_b[7])=K(P,P')=K(\gamma^{1/6},\gamma'^{1/6})$.

\begin{lem}In all cases, $\gamma$ and~$\gamma'$ are independent in
  $K^*/K^*{}^2$.  They are independent in $K^*/K^*{}^3$ unless
  $b=2\cdot7^2\cdot c^3$ with~$c\in\Q^*{}^3$, in which case
  $\gamma\gamma'=(-42c)^3$ and they generate a subgroup of $K^*/K^*{}^3$
  of order~$3$.

In the first case we have $\Gal(L/K)\cong(\Z/6\Z)^2$.  In the second
case, $\Gal(L/K)\cong(\Z/6\Z)\times(\Z/2\Z)$.
\end{lem}
\begin{proof}
Elementary.
\end{proof}
Note that $b=98c^3\implies -28/b=98(-7c)^3$, so the situation is
symmetric between $E_b$ and $E_{-28/b}$.  In this case, the curves
$E_b$ and $E_{-28/b}$ are not just sextic twists but quadratic twists
by $-7$, since $-28/b^2=(-7)^3/c^6$.

\section{Proof: generic case}

To start with, we assume that we are not in the special case, so that
$\Gal(L/K)\cong(\Z/6\Z)^2$.  We will define generating
automorphisms~$\sigma_1$, $\sigma_2\in\Gal(L/K)$, each of order~$6$, together
with an automorphism~$\tau$ of order~$2$ in $\Gal(L/\Q)$ such that
$\Gal(L/\Q)=\left<\sigma_1,\sigma_2,\tau\right>$.

Fix one value of $\gamma^{1/6}$ arbitrarily.  Let $\tau\in\Gal(L/\Q)$
be an element of order~$2$ restricting to the non-trivial element of
$\Gal(K/\Q)$.  Then $(\tau(\gamma^{1/6}))^6=\tau(\gamma)=\gamma'$, so
we may fix the value of $\gamma'^{1/6}$ by setting
$\gamma'^{1/6}=\tau(\gamma)$.  Now by independence of $\gamma^{1/6}$
and $\gamma'^{1/6}$ there exists $\sigma_1\in\Gal(L/K)$ fixing
$\gamma'^{1/6}$ and mapping $\gamma^{1/6}\mapsto-w\gamma^{1/6}$ (recall
that $-w$ is a primitive $6$th root of unity) and also
$\sigma_2\in\Gal(L/K)$ fixing $\gamma^{1/6}$ and mapping
$\gamma'^{1/6}\mapsto-w^2\gamma'^{1/6}$.

Both $\sigma_1$ and~$\sigma_2$ multiply $(\gamma/\gamma')^{1/6}$ by
the same $6$th root of unity, $-w$.  Note that
$\gamma/\gamma'=-\pi/\pibar$; one may check that
$K((\gamma/\gamma')^{1/6})=K(\zeta_7)$, and that both $\sigma_1$ and
$\sigma_2$ map $\zeta_7$ to $\zeta_7^3$.  (The image must be either
$\zeta_7^3$ or $\zeta_7^5$ since $3$ and $5$ are the primitive roots
modulo~$7$.)

Recall that $P=(\alpha^{1/3},\beta^{1/2})\in E_b(K(\gamma^{1/6}))$ is
fixed by our choice of $\gamma^{1/6}$.  We similarly have
$P'=(\alpha'^{1/3},\beta'^{1/2})\in E_b(K(\gamma'^{1/6}))$, and~$\tau$
interchanges these.  Hence we have
\[
   \rho_{E_b,7}(\tau) = \begin{pmatrix}0&1\\1&0   \end{pmatrix}.
\]


Now $\sigma_1$ fixes~$P'$, while one may see that $\sigma_1(P)=3P$
(for example the standard formulas for adding points show that
$-2P=(w^{-5}\alpha^{1/3},\beta^{1/2})=\sigma_1^5(P)$); this also
follows from the action on the $7$th roots of unity noted above, which
implies that the determinant of the matrix giving the action of
$\sigma_1$ on $\left<P,P'\right>$ is $3\pmod7$.  This matrix is
therefore
\[
   \rho_{E_b,7}(\sigma_1) = \begin{pmatrix}3&0\\0&1   \end{pmatrix},
\]
and similarly
\[
   \rho_{E_b,7}(\sigma_2) = \begin{pmatrix}1&0\\0&3   \end{pmatrix}.
\]

Now we consider the effect of twisting~$E_b$ by $u=-28/b^2$.
Replacing $b$ by $ub=-28/b$ changes $\gamma$ into $(2\r3)^6/\gamma'$.
In fact we started out by looking for a sextic twist~$u$ which induces
an isomorphism on the $7$-torsion modules, which must have the effect
of replacing $\gamma$ by $\gamma^{\pm1}$ or~$\gamma'^{\pm1}$, modulo
$6$th powers.  The only solutions with $u\in\Q^*$ are $u=-27$ and
$u=-28/b^2$.

Hence the twist leaves the $7$-division field
$K(\gamma^{1/6},\gamma'^{1/6})$ unchanged, which is certainly
necessary.

The twisting map~$\tw$ takes $(x,y)\in E_b$ to $(u^{1/3}x,u^{1/2}y)\in
E_{bu}$.  Again we need to fix the roots.  We set
$u^{1/6}=2\r3/(\gamma^{1/6}\gamma'^{1/6})$, then
$u^{1/3}=-12/(\gamma^{1/3}\gamma'^{1/3})$ and
$u^{1/2}=8\r3^3/(\gamma^{1/2}\gamma'^{1/2})$.  Hence
\[
 \tw(P) = (u^{1/3}\alpha^{1/3},u^{1/2}\alpha^{1/2}) =
(4\r3/\gamma'^{1/3},-4w\r3^3/\gamma'^{1/2}),
\]
which is fixed by~$\sigma_1$ and taken to $3$ times itself
by~$\sigma_2$.  Similarly, $\sigma_2$ fixes $\tw(P')$ and
multiples~$\tw(P)$ by~$3$.

Now $\tau$ fixes $u^{1/3}$ and negates $u^{1/2}$. A simple calculation
then gives $\tau(\tw(P)) = -\tw(P')$.

Now define $\phi: E_b[7] \to E_{bu}[7]$ by $P\mapsto\tw(P')$,
$P'\mapsto-\tw(P)$.  Then $\phi$ commutes with the action of
$\sigma_1$, $\sigma_2$ and~$\tau$, so is an isomorphism of
$G_\Q$-modules.  It is also symplectic:  denoting the Weil pairing by
$e_7$ we have
\[
e_7(\phi(P),\phi(P')) = e_7(\tw(P'),-\tw(P)) = e_7(P',-P) =
e_7(P',P)^{-1} = e_7(P,P'),
\]
where we have used the fact that $\tw$, being an isomorphism defined
over $\overline{\Q}$ preserves the Weil pairing.

This completes the proof that $E_b[7]$ and $E_{-28/b}[7]$ are
isomorphic as symplectic $G_{\Q}$-modules in the case where $b\notin
2\cdot7^2\cdot \Q^*{}^3$.


\begin{rem}
The fixed field of $\sigma_1\sigma_2$ is $K(\zeta_7)=\Q(\zeta_{21})$.
One may check that this is the field of definition of the full set of
$7$-isogenies of $E_b$: this is consistent with $\sigma_1\sigma_2$
having trivial image in $\PGL(2,\F_7)$.
\end{rem}

\section{Proof: special case}

Now we take $u=-7^3$ so the sextic twist by~$u$ is just the quadratic
twist by $-7$.  We fix $\gamma^{1/6}$ and $\gamma'^{1/6}$ with
$\tau(\gamma^{1/6})=\gamma'^{1/6}$ as before.

Since $\gamma\gamma'\in\Q^*{}^3$, any $\sigma\in\Gal(L/K)$
maps~$\gamma^{1/6}$ to $\zeta_6^r\gamma^{1/6}$ and~$\gamma'^{1/6}$ to
$\zeta_6^s\gamma^{1/6}$ where $(r,s)\in(\Z/6\Z)^2$ satisfy
$r+s\equiv0\pmod3$.  (Here $\zeta_6=-w$.) So $\Gal(L/K)$ is isomorphic
to a subgroup of index~$3$ in~$(\Z/6\Z)^2$.  As generators we may take
$\sigma_1$ of order~$6$ mapping to $(1,-1)$, and $\sigma_2$ of
order~$2$ mapping to $(0,3)$.  Then the computations carried out for
the generic case give
\[
   \rho_{E_b,7}(\sigma_1) = \begin{pmatrix}3&0\\0&3   \end{pmatrix},
\]
and
\[
   \rho_{E_b,7}(\sigma_2) = \begin{pmatrix}1&0\\0&-1   \end{pmatrix}.
\]

The twist map takes $(x,y)\mapsto(-7x,\sqrt{-7}^3y)$ and we find that
all three of $\sigma_1$, $\sigma_2$ and~$\tau$ map $\sqrt{-7}$ to
$-\sqrt{-7}$.  For this one may use the classical formula
\[
   \sqrt{-7} = \sum_{k\pmod7}\legendre{k}{7}\zeta_7^k.
\]
The same definition of~$\phi$ as in the generic case gives a
symplectic Galois isomorphism here, completing the proof.

\bibliographystyle{amsplain}
\bibliography{refs}

\end{document}
\endinput
